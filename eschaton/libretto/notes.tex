\documentclass[12pt]{article}
\usepackage{fontspec}
\usepackage[utf8]{inputenc}
\setmainfont{Bodoni 72 Book}
\usepackage[paperwidth=11in,paperheight=17in,margin=1in,headheight=0.0in,footskip=0.5in,includehead,includefoot,portrait]{geometry}
\usepackage[absolute]{textpos}
\TPGrid[0.5in, 0.25in]{23}{24}
\parindent=0pt
\parskip=12pt
\usepackage{nopageno}
\usepackage{graphicx}
\graphicspath{ {./images/} }
\usepackage{amsmath}
\usepackage{tikz}
\newcommand*\circled[1]{\tikz[baseline=(char.base)]{
            \node[shape=circle,draw,inner sep=1pt] (char) {#1};}}

\begin{document}

\vspace*{22\baselineskip}

\begingroup
\begin{center}
\huge NOTES FOR THE INTERPRETERS
\end{center}
\endgroup

\begingroup
\textbf{Forces:} This work is to be interpreted by five actors, each made distinct in the text by color. The capitalized text in red is to be prerecorded and played back in dialogue during performance. 
\endgroup

\begingroup
\textbf{Interpretation: 1.) Line breaking} indicates the flow of semantic ideas. Unbroken lines, whether they are shared by several actors or spoken by an actor alone, are read as a continuous thought. Broken lines indicate a brief pause; not a full beat, but a breath. \textbf{2.) Text enclosed in parentheses} is not spoken, but the actions described within are staged, as the text is projected before the audience. The projections persist for the amount of time it would reasonably take for the included text to be read aloud.
\endgroup

\end{document}